\documentclass[
11pt, % The default document font size, options: 10pt, 11pt, 12pt
%oneside, % Two side (alternating margins) for binding by default, uncomment to switch to one side
english, % ngerman for German
singlespacing, % Single line spacing, alternatives: onehalfspacing or doublespacing
%draft, % Uncomment to enable draft mode (no pictures, no links, overfull hboxes indicated)
%nolistspacing, % If the document is onehalfspacing or doublespacing, uncomment this to set spacing in lists to single
%liststotoc, % Uncomment to add the list of figures/tables/etc to the table of contents
%toctotoc, % Uncomment to add the main table of contents to the table of contents
%parskip, % Uncomment to add space between paragraphs
%nohyperref, % Uncomment to not load the hyperref package
headsepline, % Uncomment to get a line under the header
%chapterinoneline, % Uncomment to place the chapter title next to the number on one line
%consistentlayout, % Uncomment to change the layout of the declaration, abstract and acknowledgements pages to match the default layout
]{StyleSheet} % The class file specifying the document structure

\usepackage[utf8]{inputenc} % Required for inputting international characters
\usepackage[T1]{fontenc} % Output font encoding for international characters
\usepackage{palatino} % Use the Palatino font by default
\usepackage[backend=bibtex,style=authoryear,natbib=true]{biblatex} % Use the bibtex backend with the authoryear citation style (which resembles APA)
% Refer Bibliography Resource
\usepackage[autostyle=true]{csquotes} % Required to generate language-dependent quotes in the bibliography
\usepackage[figuresleft]{rotating}

\addbibresource{example.bib} % The filename of the bibliography

\newcommand{\drawingThree}[2]{
\begin{figure}[!ht]
    \centering
    \includegraphics[width=0.45\textwidth]{#1}
    \caption{#2}
\end{figure}
}

\newcommand{\drawing}[2]{
\begin{sidewaysfigure}
    \centering
    \includegraphics[width=0.85\textwidth]{#1}
    \caption{#2}
\end{sidewaysfigure}
}

\newcommand{\drawingTwo}[2]{
\begin{figure}[!ht]
    \centering
    \includegraphics[width=0.75\textwidth]{#1}
    \caption{#2}
\end{figure}
}

\geometry{
	paper=letterpaper, % Change to letterpaper for US letter
	inner=2.5cm, % Inner margin
	outer=3.8cm, % Outer margin
	bindingoffset=.5cm, % Binding offset
	top=1.5cm, % Top margin
	bottom=1.5cm, % Bottom margin
	%showframe, % Uncomment to show how the type block is set on the page
}

\thesistitle{GT Pioneer} 
% Your thesis title, this is used in the title and abstract, print it elsewhere with \ttitle

\supervisor{Dr. Raghuram \textsc{Pucha}} 
% Your supervisor's name, this is used in the title page, print it elsewhere with \supname

\examiner{} 
% Your examiner's name, this is not currently used anywhere in the template, print it elsewhere with \examname

\degree{ME1770} 
% Your degree name, this is used in the title page and abstract, print it elsewhere with \degreename

\author{Vishakh \textsc{Kumar}} 
% Your name, this is used in the title page and abstract, print it elsewhere with \authorname

\addresses{} 
% Your address, this is not currently used anywhere in the template, print it elsewhere with \addressname

\subject{ME 1770} 
% Your subject area, this is not currently used anywhere in the template, print it elsewhere with \subjectname

\keywords{Georgia Tech, Mars Rover, Pioneer} 
% Keywords for your thesis, this is not currently used anywhere in the template, print it elsewhere with \keywordnames

\university{\href{http://www.gatech.edu}{Georgia Institute of Technology}}

\department{\href{http://me.gatech.edu}{George W Woodruff School of Mechanical Engineering}}
% Your department's name and URL, this is used in the title page and abstract, print it elsewhere with \deptname

\group{\href{https://github.com/vishakhkumar/ME1770}{Group X}}

\faculty{\href{http://faculty.university.com}{Faculty Name}}
% Your faculty's name and URL, this is used in the title page and abstract, print it elsewhere with \facname

\AtBeginDocument{
\hypersetup{pdftitle=\ttitle} % Set the PDF's title to your title
\hypersetup{pdfauthor=\authorname} % Set the PDF's author to your name
\hypersetup{pdfkeywords=\keywordnames} % Set the PDF's keywords to your keywords
}

\begin{document}

\frontmatter % Use roman page numbering style (i, ii, iii, iv...) for the pre-content pages
\pagestyle{plain} % Default to the plain heading style until the thesis style is called for the body content

\input{a-0-1-TitlePage/TitlePage.tex}

\mainmatter % Begin numeric (1,2,3...) page numbering
\pagestyle{thesis} % Return the page headers back to the "thesis" style

\chapter{Project Ideation}

\section{Project Proposal}
\subsection{Description of Product / Structure: Describe the creative ideation and what is new?}

Our product is a Mars capable ATV. We began with the idea of the standard ATV, coupled with the idea of a manned Mars rover. By combining these two concepts, we were able to create a more agile vehicle capable of handling Mars’ low gravity and dusty environment. The combination of a pressurized capsule in an off-road vehicle can be challenging but the benefits would be immense in creating robust vehicles for a manned colony on Mars.

\subsection{Description of subsystem and allocation for each member}

\begin{center}
\begin{tabular}{lll}
\hline
Subsystem & Description & Member\\
\hline
Orbital Deployment & Circular parachute and coiled spring shocks. & Vishakh Kumar\\
Grabbers & Pivoting arm with ball socket and grabbing hands. & (S)\\
Suspension & Coil spring shocks, double A-frame suspension and tire rods. & (H)\\
Chassis & Triangular truss support frame. & (S)\\
Tires & Cylindrical tires with embossed treads. & (F)\\
Controls & Joystick,Displays, Plexiglas encased w/ rectangular control panel. & Vishakh Kumar\\
Cockpit & Oblong shaped cockpit & Vishakh Kumar\\
Powertrain & Circular Motor with chain drive to rear axle with rear diff. & (H)\\
Charging & Rectangular solar cells on roof. & Vishakh Kumar\\
Science/Storage & Large prismed storage area in back of ATV. & (F)\\
Communication System & Conic Satellite Dish. & (R)\\
Lighting & Semi-Paraboloid lights mounted on front of ATV. & (R)\\
\hline
\end{tabular}
\end{center}

\subsection{Define functionality– Design, Modeling and Assembly Complexity:}

\begin{center}
\begin{tabular}{lll}
\hline
Subsystem & Functionality & Complexity\\
\hline
Orbital Deployment & Landing Gear when ATV is dropped from orbit. & (K)\\
Grabbers & Grabs materials for data inspection. & (S)\\
Suspension & Absorbs shocks from planetary terrain. & (H)\\
Chassis & Beefy frame for surviving rough conditions. & (S)\\
Tires & Extreme grip to handle unexpected terrain. & (F)\\
Controls & Steering, cockpit, seating, etc. & (K)\\
Cockpit & Location of Controls & (K)\\
Powertrain & Electric drivetrain, differential. & (H)\\
Charging & Solar cells from roof to charge batteries behind cockpit. & (K)\\
Science/Storage & Large storage area in back of ATV to collect data/samples. & (F)\\
Communication System & Antenna to communicate with base. & (R)\\
Lighting & To maintain visibility once night falls or in sandstorms. & (R)\\
\hline
\end{tabular}
\end{center}

\subsection{Briefly explain what new functionalities (system and sub-system ) you are planning to add. How your product is different from existing products:}

This design differs from the traditional ATV because it has a improved suspension system for travel along Martian terrain. The ATV will be able to withstand orbital entry into the Martian landscape through its improved suspension and parachute for controlled descent. Additionally for increased driver visibility the pressurized cabin is built with GT-Superglass® which has the material strength of hardened steel and the weight of titanium. With this glass our vehicle will be able to withstand sandstorms containing heavy debris.

\subsection{Picture of  the Proposed System (or Similar System): (please include a reference if you are using pictures from internet). You can also include conceptual sketch.}
\includegraphics{c-1-Images/Planetside.png}
(Daybreak Games: Planetside 2 ANT Vehicle Concept)

(\url{https://grabcad.com/library/baja-atv-1} BAJA SAE India Team)
\includegraphics{c-1-Images/BAJA.jpeg}


\clearpage
 \section{Project Management}
 

Insert spiel about ProjectManagement.

\subsection{PartDistribution}
\input{a-1-1-ProjectIdeation/b-2-ProjectManagement/c-1-PartDistribution/PartDistribution.tex}

\subsection{Planning}
\input{a-1-1-ProjectIdeation/b-2-ProjectManagement/c-2-Planning/Planning.tex}

\subsection{Timeline}
\input{a-1-1-ProjectIdeation/b-2-ProjectManagement/c-3-Timeline/Timeline.tex}




\chapter{Preliminary Design}

Insert spiel about PreliminaryDesign.

\section{Conceptual Sketches}
Our project was fairly ambitious in that we combined two very different worlds - the rough and tumble world of off-road vehicles and the pressurized environments of space vehicles. Conceptual drawings were invaluable in sketching out a basic idea of what this vehicle would look like.


\section{Isometric Sketches}
After sketching out our conceptual drawings and allocating tasks between team members, we then proceded to create isometric drawings of each assembly and the top level subassemblies.
This helped us refine our ideas about what our parts would look like and how we could improve them.
As our product was fairly complicated, we also had the benefit of improving our drawing skills - more than a few parts had interesting features that were a challenge to draw.

\subsection{Asimm}
% \drawing{/a-1-2-PreliminaryDesign/b-1-Conceptual/Assim}{Hirani, Asimm: }
\drawingThree{/a-1-2-PreliminaryDesign/b-1-Conceptual/Asimm/IMG_1809.JPG}{Hirani, Asimm: Suspension}
\drawingThree{/a-1-2-PreliminaryDesign/b-1-Conceptual/Asimm/IMG_1810.JPG}{Hirani, Asimm: Powertrain}
\drawingThree{/a-1-2-PreliminaryDesign/b-1-Conceptual/Asimm/IMG_1815.JPG}{Hirani, Asimm: Scientiic Storage}


\subsection{Auston}
\drawingThree{/a-1-2-PreliminaryDesign/b-1-Conceptual/Auston}{Ferrarer, Auston: }


\subsection{Juan}
\IsometricDrawing{/a-1-2-PreliminaryDesign/b-3-Multiview/Juan/20161206_022025_resized.jpg}{Rodriguez, Juan: Antenna}
\IsometricDrawing{/a-1-2-PreliminaryDesign/b-3-Multiview/Juan/20161206_022035_resized.jpg}{Rodriguez, Juan: Antenna}
\IsometricDrawing{/a-1-2-PreliminaryDesign/b-3-Multiview/Juan/20161206_022045_resized.jpg}{Rodriguez, Juan: Antenna}
\IsometricDrawing{/a-1-2-PreliminaryDesign/b-3-Multiview/Juan/20161206_022249_resized.jpg}{Rodriguez, Juan: Chassis}
\IsometricDrawing{/a-1-2-PreliminaryDesign/b-3-Multiview/Juan/20161206_022255_resized.jpg}{Rodriguez, Juan: Chassis}
\IsometricDrawing{/a-1-2-PreliminaryDesign/b-3-Multiview/Juan/20161206_022314_resized.jpg}{Rodriguez, Juan: Chassis}


\subsection{Justin}
\drawingThree{/a-1-2-PreliminaryDesign/b-1-Conceptual/Assim}{Sackett, Justin: }


\subsection{Vishakh}
\drawingThree{/a-1-2-PreliminaryDesign/b-1-Conceptual/Vishakh}{Kumar, Vishakh: }



\section{Multiview Sketches}
\subsection{Multiview Drawings}

\subsubsection{Vishakh Kumar}

\subsubsection{Justin Sackett}

\subsubsection{Asimm Hirani}

\subsubsection{Juan Rodriguez}

\subsubsection{Auston Ferrarer}




\chapter{Detail Design}
\newcommand{\DetailDesign}[1]{
\section{#1}
\input{a-1-3-DetailDesign/b-#1/#1.tex}
}

\newcommand{\DetailDesignDrawing}[2]{
\begin{figure}[!ht]
    \centering
    \includegraphics[width=0.75\textwidth]{#1}
    \caption{#2}
\end{figure}
}

\DetailDesign{Seat}
\clearpage
\DetailDesign{Bed}
\clearpage
\DetailDesign{DoorandExterior}
\clearpage
\DetailDesign{Cockpit}
\clearpage
\DetailDesign{Joystick}
\clearpage
\DetailDesign{Dial}
\clearpage
%\DetailDesign{Lights}
%\clearpage
\DetailDesign{CabinetDoor}
\clearpage
%\DetailDesign{Grabbers}
%\clearpage
\DetailDesign{EmergencySwitch}
\clearpage
\DetailDesign{MechanicalDisplay}
\clearpage
\DetailDesign{3DPrinter}
\clearpage
\DetailDesign{Chassis}
\clearpage
\DetailDesign{Helmet}
\clearpage
\DetailDesign{Suspension}
\clearpage
\DetailDesign{Antenna}



\chapter{ManufacturingWorkingDrawing}
Insert spiel about ManufacturingWorkingDrawing.

\section{WorkingDrawing}


\chapter{WorkingDrawing}

Insert spiel about DetailDesign.

\section{Antenna}
\WorkingDrawing{a-1-4-ManufacturingWorkingDrawing/b-1-WorkingDrawing/c-Antenna/Rodriguez_Juan_Signal_Bar.JPG}{Signal Bar}
\WorkingDrawing{a-1-4-ManufacturingWorkingDrawing/b-1-WorkingDrawing/c-Antenna/Rodriguez_Juan_Swivel_link.JPG}{Swivel Link}
\WorkingDrawing{a-1-4-ManufacturingWorkingDrawing/b-1-WorkingDrawing/c-Antenna/Rodriguez_Juan_antenna_assembly.JPG}{Antenna Assembly}
\WorkingDrawing{a-1-4-ManufacturingWorkingDrawing/b-1-WorkingDrawing/c-Antenna/Rodriguez_Juan_antenna_base.JPG}{Antenna Base}
\WorkingDrawing{a-1-4-ManufacturingWorkingDrawing/b-1-WorkingDrawing/c-Antenna/Rodriguez_Juan_antenna_exploded.JPG}{Antenna Exploded}
\WorkingDrawing{a-1-4-ManufacturingWorkingDrawing/b-1-WorkingDrawing/c-Antenna/Rodriguez_Juan_antenna_support.JPG}{Antenna Support}
\WorkingDrawing{a-1-4-ManufacturingWorkingDrawing/b-1-WorkingDrawing/c-Antenna/Rodriguez_Juan_crossbar.JPG}{Crossbar}
\WorkingDrawing{a-1-4-ManufacturingWorkingDrawing/b-1-WorkingDrawing/c-Antenna/Rodriguez_Juan_pin_a.JPG}{Pin A}
\WorkingDrawing{a-1-4-ManufacturingWorkingDrawing/b-1-WorkingDrawing/c-Antenna/{Rodriguez_Juan_pin_b.JPG}{Pin B}
\WorkingDrawing{a-1-4-ManufacturingWorkingDrawing/b-1-WorkingDrawing/c-Antenna/Rodriguez_Juan_real_antenna.JPG}{Real Antenna}
\WorkingDrawing{a-1-4-ManufacturingWorkingDrawing/b-1-WorkingDrawing/c-Antenna/Rodriguez_Juan_receiver.JPG}{Receiver}



\section{AssemblyInstructionManual}
% \WorkingDrawing{Antenna}
% \WorkingDrawing{Chassis}
% \WorkingDrawing{Grabber}
% \WorkingDrawing{Suspension}
% \WorkingDrawing{Cockpit}
% \WorkingDrawing{Joystick}
% \WorkingDrawing{MechanicalDisplay}

\section{Working Drawings}
A product cannot be counted as finished unless there are plans to manufacture that product. While the plans for our product is definitely beyond the ability of a student run organization (or small countries), 
we've added working drawings to highlight important features and dimensions of our work.

\newcommand{\WorkingDrawing}[1]{
 \subsection{#1}
 \input{a-1-4-ManufacturingWorkingDrawing/b-1-WorkingDrawing/c-#1/#1.tex}
 }


\section{ExplodedView}


\clearpage
\section{ExplodedView}

Insert spiel about ExplodedView.
\clearpage

\subsection{Antenna}
 \WorkingDrawing{a-1-4-ManufacturingWorkingDrawing/b-1-WorkingDrawing/c-Antenna/Rodriguez_Juan_Signal_Bar.JPG}{Signal Bar}
\WorkingDrawing{a-1-4-ManufacturingWorkingDrawing/b-1-WorkingDrawing/c-Antenna/Rodriguez_Juan_Swivel_link.JPG}{Swivel Link}
\WorkingDrawing{a-1-4-ManufacturingWorkingDrawing/b-1-WorkingDrawing/c-Antenna/Rodriguez_Juan_antenna_assembly.JPG}{Antenna Assembly}
\WorkingDrawing{a-1-4-ManufacturingWorkingDrawing/b-1-WorkingDrawing/c-Antenna/Rodriguez_Juan_antenna_base.JPG}{Antenna Base}
\WorkingDrawing{a-1-4-ManufacturingWorkingDrawing/b-1-WorkingDrawing/c-Antenna/Rodriguez_Juan_antenna_exploded.JPG}{Antenna Exploded}
\WorkingDrawing{a-1-4-ManufacturingWorkingDrawing/b-1-WorkingDrawing/c-Antenna/Rodriguez_Juan_antenna_support.JPG}{Antenna Support}
\WorkingDrawing{a-1-4-ManufacturingWorkingDrawing/b-1-WorkingDrawing/c-Antenna/Rodriguez_Juan_crossbar.JPG}{Crossbar}
\WorkingDrawing{a-1-4-ManufacturingWorkingDrawing/b-1-WorkingDrawing/c-Antenna/Rodriguez_Juan_pin_a.JPG}{Pin A}
\WorkingDrawing{a-1-4-ManufacturingWorkingDrawing/b-1-WorkingDrawing/c-Antenna/{Rodriguez_Juan_pin_b.JPG}{Pin B}
\WorkingDrawing{a-1-4-ManufacturingWorkingDrawing/b-1-WorkingDrawing/c-Antenna/Rodriguez_Juan_real_antenna.JPG}{Real Antenna}
\WorkingDrawing{a-1-4-ManufacturingWorkingDrawing/b-1-WorkingDrawing/c-Antenna/Rodriguez_Juan_receiver.JPG}{Receiver}

\clearpage

\subsection{Lights}
 \newcommand{\DetailDesignLights}[2]{\drawing{/a-1-3-DetailDesign/b-Lights/#1}{Sackett, Justin: #2}
\DetailDesignDrawing{/a-1-3-DetailDesign/b-Lights/LIGHTBASE.JPG}{\justin Light Base}
\DetailDesignDrawing{/a-1-3-DetailDesign/b-Lights/LIGHTSCREWS.JPG}{\justin Light Screws}
\DetailDesignDrawing{/a-1-3-DetailDesign/b-Lights/LED_Work_light_27w.JPG}{\justin LED Work Light}

\clearpage


\section{PartsList}
\newcommand{\PartListDrawing}[2]{
\begin{figure}[!ht]
    \centering
    \includegraphics[height=0.85\textheight]{#1}
    \caption{#2}
\end{figure}
}

\subsection{Antenna}
\WorkingDrawing{a-1-4-ManufacturingWorkingDrawing/b-1-WorkingDrawing/c-Antenna/Rodriguez_Juan_Signal_Bar.JPG}{Signal Bar}
\WorkingDrawing{a-1-4-ManufacturingWorkingDrawing/b-1-WorkingDrawing/c-Antenna/Rodriguez_Juan_Swivel_link.JPG}{Swivel Link}
\WorkingDrawing{a-1-4-ManufacturingWorkingDrawing/b-1-WorkingDrawing/c-Antenna/Rodriguez_Juan_antenna_assembly.JPG}{Antenna Assembly}
\WorkingDrawing{a-1-4-ManufacturingWorkingDrawing/b-1-WorkingDrawing/c-Antenna/Rodriguez_Juan_antenna_base.JPG}{Antenna Base}
\WorkingDrawing{a-1-4-ManufacturingWorkingDrawing/b-1-WorkingDrawing/c-Antenna/Rodriguez_Juan_antenna_exploded.JPG}{Antenna Exploded}
\WorkingDrawing{a-1-4-ManufacturingWorkingDrawing/b-1-WorkingDrawing/c-Antenna/Rodriguez_Juan_antenna_support.JPG}{Antenna Support}
\WorkingDrawing{a-1-4-ManufacturingWorkingDrawing/b-1-WorkingDrawing/c-Antenna/Rodriguez_Juan_crossbar.JPG}{Crossbar}
\WorkingDrawing{a-1-4-ManufacturingWorkingDrawing/b-1-WorkingDrawing/c-Antenna/Rodriguez_Juan_pin_a.JPG}{Pin A}
\WorkingDrawing{a-1-4-ManufacturingWorkingDrawing/b-1-WorkingDrawing/c-Antenna/{Rodriguez_Juan_pin_b.JPG}{Pin B}
\WorkingDrawing{a-1-4-ManufacturingWorkingDrawing/b-1-WorkingDrawing/c-Antenna/Rodriguez_Juan_real_antenna.JPG}{Real Antenna}
\WorkingDrawing{a-1-4-ManufacturingWorkingDrawing/b-1-WorkingDrawing/c-Antenna/Rodriguez_Juan_receiver.JPG}{Receiver}


\subsection{Lights}
\newcommand{\DetailDesignLights}[2]{\drawing{/a-1-3-DetailDesign/b-Lights/#1}{Sackett, Justin: #2}
\DetailDesignDrawing{/a-1-3-DetailDesign/b-Lights/LIGHTBASE.JPG}{\justin Light Base}
\DetailDesignDrawing{/a-1-3-DetailDesign/b-Lights/LIGHTSCREWS.JPG}{\justin Light Screws}
\DetailDesignDrawing{/a-1-3-DetailDesign/b-Lights/LED_Work_light_27w.JPG}{\justin LED Work Light}


\subsection{Suspension}
\AssemblyManualStep{1}{a-1-4-ManufacturingWorkingDrawing/b-2-AssemblyInstructionManual/c-Suspension/step1.PNG}{
Assemble Full Wheel by taking wheel and putting into Tire.
}{\asimm Assembly Step 1}

\AssemblyManualStep{2}{a-1-4-ManufacturingWorkingDrawing/b-2-AssemblyInstructionManual/c-Suspension/step2.PNG}{
Insert Hub into Wheel
}{\asimm Assembly Step 2}

\AssemblyManualStep{3}{a-1-4-ManufacturingWorkingDrawing/b-2-AssemblyInstructionManual/c-Suspension/step3.PNG}{
Mount Brake Disk and Brake Caliper onto Hub
}{\asimm Assembly Step 3}

\AssemblyManualStep{4}{a-1-4-ManufacturingWorkingDrawing/b-2-AssemblyInstructionManual/c-Suspension/step4.PNG}{
Mount hub and wheel assembly onto upright
}{\asimm Assembly Step 4}

\AssemblyManualStepTwo{5}{a-1-4-ManufacturingWorkingDrawing/b-2-AssemblyInstructionManual/c-Suspension/step5.PNG}{
Mount lower control arms and upper controls arms onto frame. Mount coilover onto LCA. Mount upright between control arms. Mirror assembly accross the frame. 
}{\asimm Assembly Step 5}


\subsection{Grabbers}
\drawing{/a-1-3-DetailDesign/b-1-WorkingDrawing/c-Antenna/Rodriguez_Juan_Signal_Bar.JPG}{Rodriguez, Juan: Chasis Render}


\subsection{Emergency Button}
\ExplodedViewDrawing{a-1-4-ManufacturingWorkingDrawing/b-3-ExplodedView/c-EmergencyButton/EmergencyButton.JPG}{\vishakh Exploded View of Emergency Button}


\subsection{Mechanical Display}
\DetailDesignDrawing{a-1-3-DetailDesign/b-MechanicalDisplay/MechanicalDisplay.png}{\vishakh Mechanical Display View 1}
\DetailDesignDrawing{a-1-3-DetailDesign/b-MechanicalDisplay/MechanicalDisplay1.png}{\vishakh Mechanical DisplayView 2}
\DetailDesignDrawing{a-1-3-DetailDesign/b-MechanicalDisplay/MechanicalDisplay2.png}{\vishakh Mechanical Display View 3}
%\DetailDesignDrawing{a-1-3-DetailDesign/b-MechanicalDisplay/MechanicalDisplay3.png}{\vishakh Mechanical Display View 4}





\chapter{CheckForFunctionality}
Insert spiel about CheckForFunctionality.

\section{SwivelLight}
\input{a-1-5-CheckForFunctionality/b-SwivelLight/SwivelLight.tex}



\chapter{SummaryAndConcludingRemarks}
Insert spiel about SummaryAndConcludingRemarks.

\section{MetObjectiveAndGoal}
\input{a-1-5-SummaryAndConcludingRemarks/b-1-MetObjectiveAndGoal/MetObjectiveAndGoal.tex}

\section{CourseComment}
\input{a-1-5-SummaryAndConcludingRemarks/b-2-CourseComment/CourseComment.tex}

\section{TeamExperience}
\input{a-1-5-SummaryAndConcludingRemarks/b-3-TeamExperience/TeamExperience.tex}

\section{CourseSuggestions}
The only suggestion we have about this project is to perhaps move the deadline for the group project a couple weeks prior to exam week to avoid conflicts amongst classes.  This also helps group members be at group meetings as group members may need to study for another class.



\appendix


\chapter{Meeting Minutes}
As a team, we used meetings to inform each other about our individual progress rather than use the time to develop our models.
As such our meetings were often short and straight forward.

We've added an abridged version of our meeting notes. For a more complete version of the meeting notes, please refer the notes and log stored at \url{https://www.github.com/vishakhkumar.ME1770}

\newcolumntype{L}{>{\RaggedRight\arraybackslash}X} % ragged-right version of "X"
\begin{table}[t]
\caption{Meeting Minutes}
\begin{center}
\begin{tabularx}{\textwidth}{|L|L|L|}

\hline
DATE & MINUTES & DELIVERABLES FOR NEXT MEETING \\
\hline
\hline
& & \\
11/3/16 & 
Decide on Project Ideation and decide on division of labour.
&
3 sketches per part 
\begin{itemize}
\item Conceptual
\item Isometric
\item Multiviews
\end{itemize}
\\
%%%%%%%%%%%%%%%%%%%%%%%%%%%%%%%%%%%%%%%%%%%%%%%%%%%
\hline
& & \\
%%%%%%%%%%%%%%%%%%%%%%%%%%%%%%%%%%%%%%%%%%%%%%%%%%%
11/10/16 &
 Sketches of all parts due and reviewed by Dr. Pucha
&
\begin{itemize}
\item Begin modelling in Solidworks. Create basic subassemblies.
\item Communicate with adjoining subassemblies to discuss attachment structures.
\end{itemize}
\\
%%%%%%%%%%%%%%%%%%%%%%%%%%%%%%%%%%%%%%%%%%%%%%%%%%%
\hline
& & \\
%%%%%%%%%%%%%%%%%%%%%%%%%%%%%%%%%%%%%%%%%%%%%%%%%%%
11/17/16 & 
Finish all subassemblies. Request help from other members should such a need arise.
&
Finish final assemblies by Sunday.
\\
%%%%%%%%%%%%%%%%%%%%%%%%%%%%%%%%%%%%%%%%%%%%%%%%%%%
\hline
& & \\
%%%%%%%%%%%%%%%%%%%%%%%%%%%%%%%%%%%%%%%%%%%%%%%%%%%
11/24/16 & 
Combine all subassemblies into final assembly.
&
Submit all subassemblies to Juan and Vishakh.
\\
%%%%%%%%%%%%%%%%%%%%%%%%%%%%%%%%%%%%%%%%%%%%%%%%%%%
\hline
& & \\
%%%%%%%%%%%%%%%%%%%%%%%%%%%%%%%%%%%%%%%%%%%%%%%%%%%
12/1/16 & 
Finish solidworks drawings, renders and animation.
&
Each team member must deliver (for each subassembly and part): 
\begin{itemize}
\item Solidworks Engineering Drawings
\item Part Renders
\item Animation
\end{itemize}
\\
%%%%%%%%%%%%%%%%%%%%%%%%%%%%%%%%%%%%%%%%%%%%%%%%%%%
\hline
& & \\
%%%%%%%%%%%%%%%%%%%%%%%%%%%%%%%%%%%%%%%%%%%%%%%%%%%
12/8/16 & 
Finish Final Report
&
Push final report to github and print before Friday.
\\
%%%%%%%%%%%%%%%%%%%%%%%%%%%%%%%%%%%%%%%%%%%%%%%%%%%
\hline
%%%%%%%%%%%%%%%%%%%%%%%%%%%%%%%%%%%%%%%%%%%%%%%%%%%




\end{tabularx}
\end{center}
\end{table}


\end{document}
