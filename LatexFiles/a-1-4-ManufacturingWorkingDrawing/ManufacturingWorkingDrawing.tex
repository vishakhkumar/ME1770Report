

\chapter{Manufacturing Working Drawing}
Insert spiel about ManufacturingWorkingDrawing.
\begin{itemize}
\item \ref{sec:WorkingDrawing} \textbf{Working Drawings}\\
A product cannot be counted as finished unless there are plans to manufacture that product. While the plans for our product is definitely beyond the ability of a student run organization (or small countries), 
we've added working drawings to highlight important features and dimensions of our work.

\item \ref{sec:AssemblyInstructionManual} \textbf{Assembly Instruction Manual}\\
While our product is unlikely to ever reach the manufacturing line, it's prudent to think about how products are manufactured and assembled in order to create functional products.
In that spirit, we've created assembly instructions for an intrepid student to follow should s/he ever attempt building a Mars Rover.

\item \ref{sec:ExplodedView} \textbf{Exploded View}\\

\item \ref{sec:PartList}  \textbf{Part List}\\

\clearpage
\section{Working Drawings} \label{sec:WorkingDrawing}
 \WorkingDrawing{Antenna}
\WorkingDrawing{Chassis}
\WorkingDrawing{Grabber}
\WorkingDrawing{Suspension}

\WorkingDrawing{3DPrinter}
\WorkingDrawing{Bed}
\WorkingDrawing{CabinetDoor}
\WorkingDrawing{Chassis}
\WorkingDrawing{DoorAndHinge}
\WorkingDrawing{ExteriorShell}
\WorkingDrawing{Helmet}

% \WorkingDrawing{Cockpit}
% \WorkingDrawing{Joystick}
% \WorkingDrawing{MechanicalDisplay}

\section{Working Drawings}
A product cannot be counted as finished unless there are plans to manufacture that product. While the plans for our product is definitely beyond the ability of a student run organization (or small countries), 
we've added working drawings to highlight important features and dimensions of our work.

\newcommand{\WorkingDrawing}[1]{
 \subsection{#1}
 \input{a-1-4-ManufacturingWorkingDrawing/b-1-WorkingDrawing/c-#1/#1.tex}
 }


\clearpage
 \section{Assembly Instruction Manual} \label{sec:AssemblyInstructionManual}
 %\newcommand{\AssemblyInstructionManual}[1]{
 \subsection{#1}
 \input{a-1-4-ManufacturingWorkingDrawing/b-2-AssemblyInstructionManual/c-#1/#1.tex}
 }


\newcommand{\AssemblyManualStep}[4]{
\subsubsection{Step #1}
\begin{center}
#3
\begin{figure}[!ht]
\centering
\includegraphics[height=0.70\textheight]{#2}
\caption{#4}
\end{figure}
\end{center}
\clearpage
}

\newcommand{\AssemblyManualStepTwo}[4]{
\subsubsection{Step #1}
\begin{center}
#3
\begin{figure}[!ht]
\centering
\includegraphics[width=0.90\textwidth]{#2}
\caption{#4}
\end{figure}
\end{center}
}

\AssemblyInstructionManual{Antenna}
\clearpage
\AssemblyInstructionManual{Grabber}
\clearpage
\AssemblyInstructionManual{Lights}
\clearpage
\AssemblyInstructionManual{Suspension}
\clearpage
\AssemblyInstructionManual{3DPrinter}
\clearpage
\AssemblyInstructionManual{FlexBed}
\clearpage
\AssemblyInstructionManual{Helmet}
\clearpage
\AssemblyInstructionManual{Hinge}

% \AssemblyInstructionManual{Cockpit}
% \AssemblyInstructionManual{Joystick}
% \AssemblyInstructionManual{MechanicalDisplay}


\clearpage
\section{Exploded View} \label{sec:ExplodedView}
 % 

\clearpage
\section{ExplodedView}
Insert spiel about ExplodedView.

\subsection{Antenna}
<<<<<<< HEAD
\drawing{a-1-4-ManufacturingWorkingDrawing/b-3-ExplodedView/c-Antenna/AntennaAssemblyExploded.JPG}{Exploded View of Antenna Assembly}
=======
\drawingTwo{a-1-4-ManufacturingWorkingDrawing/b-3-ExplodedView/c-Antenna/AntennaAssemblyExploded.JPG}{Exploded View of Antenna Assembly}
>>>>>>> b26c8c7fe51429f8ced673c0452ed2308f1da92f


\subsection{Lights}
\newcommand{\AssemblyManualLights}[3]{
\subsubsection{Step #1}
\begin{center}
#3
\begin{figure}[!ht]
\includegraphics[width=0.85\textwidth]{/a-1-4-ManufacturingWorkingDrawing/b-2-AssemblyInstructionManual/c-Lights/#2}
\caption{Sackett, Justin: Assembly Step #1}
\begin{figure}
\end{center}
}
\AssemblyManualLights{1}{step1.jpeg}{
Place all 11 light subassemblies on top of the 11 pre manufactured holes in the light base.
}
\AssemblyManualLights{2}{step2.jpeg}{
Screw each of the 11 light screws into each of the 11 holes in order to secure the lights into the light base.
}
\AssemblyManualLights{3}{step3.jpeg}{
Complete the assembly by securely fastening each screw.
}



\clearpage
\section{Part List} \label{sec:PartList} 
% 

\section{PartsList}

Insert spiel about PartsList.

\subsection{Antenna}
<<<<<<< HEAD
\drawing{a-1-4-ManufacturingWorkingDrawing/b-3-ExplodedView/c-Antenna/AntennaAssemblyExploded.JPG}{Exploded View of Antenna Assembly}
=======
\drawingTwo{a-1-4-ManufacturingWorkingDrawing/b-3-ExplodedView/c-Antenna/AntennaAssemblyExploded.JPG}{Exploded View of Antenna Assembly}
>>>>>>> b26c8c7fe51429f8ced673c0452ed2308f1da92f


