

\chapter{Manufacturing Working Drawing}
Insert spiel about ManufacturingWorkingDrawing.
\begin{itemize}
\item \ref{sec:WorkingDrawing} \textbf{Working Drawings}\\
A product cannot be counted as finished unless there are plans to manufacture that product. While the plans for our product is definitely beyond the ability of a student run organization (or small countries), 
we've added working drawings to highlight important features and dimensions of our work.

\item \ref{sec:AssemblyInstructionManual} \textbf{Assembly Instruction Manual}\\
While our product is unlikely to ever reach the manufacturing line, it's prudent to think about how products are manufactured and assembled in order to create functional products.
In that spirit, we've created assembly instructions for an intrepid student to follow should s/he ever attempt building a Mars Rover.

\item \ref{sec:ExplodedView} \textbf{Exploded View}\\

\item \ref{sec:PartList}  \textbf{Part List}\\

\clearpage
\section{Working Drawings} \label{sec:WorkingDrawing}
 

\chapter{WorkingDrawing}

Insert spiel about DetailDesign.

\section{Antenna}
\WorkingDrawing{a-1-4-ManufacturingWorkingDrawing/b-1-WorkingDrawing/c-Antenna/Rodriguez_Juan_Signal_Bar.JPG}{Signal Bar}
\WorkingDrawing{a-1-4-ManufacturingWorkingDrawing/b-1-WorkingDrawing/c-Antenna/Rodriguez_Juan_Swivel_link.JPG}{Swivel Link}
\WorkingDrawing{a-1-4-ManufacturingWorkingDrawing/b-1-WorkingDrawing/c-Antenna/Rodriguez_Juan_antenna_assembly.JPG}{Antenna Assembly}
\WorkingDrawing{a-1-4-ManufacturingWorkingDrawing/b-1-WorkingDrawing/c-Antenna/Rodriguez_Juan_antenna_base.JPG}{Antenna Base}
\WorkingDrawing{a-1-4-ManufacturingWorkingDrawing/b-1-WorkingDrawing/c-Antenna/Rodriguez_Juan_antenna_exploded.JPG}{Antenna Exploded}
\WorkingDrawing{a-1-4-ManufacturingWorkingDrawing/b-1-WorkingDrawing/c-Antenna/Rodriguez_Juan_antenna_support.JPG}{Antenna Support}
\WorkingDrawing{a-1-4-ManufacturingWorkingDrawing/b-1-WorkingDrawing/c-Antenna/Rodriguez_Juan_crossbar.JPG}{Crossbar}
\WorkingDrawing{a-1-4-ManufacturingWorkingDrawing/b-1-WorkingDrawing/c-Antenna/Rodriguez_Juan_pin_a.JPG}{Pin A}
\WorkingDrawing{a-1-4-ManufacturingWorkingDrawing/b-1-WorkingDrawing/c-Antenna/{Rodriguez_Juan_pin_b.JPG}{Pin B}
\WorkingDrawing{a-1-4-ManufacturingWorkingDrawing/b-1-WorkingDrawing/c-Antenna/Rodriguez_Juan_real_antenna.JPG}{Real Antenna}
\WorkingDrawing{a-1-4-ManufacturingWorkingDrawing/b-1-WorkingDrawing/c-Antenna/Rodriguez_Juan_receiver.JPG}{Receiver}



\clearpage
 \section{Assembly Instruction Manual} \label{sec:AssemblyInstructionManual}
 %% \WorkingDrawing{Antenna}
% \WorkingDrawing{Chassis}
% \WorkingDrawing{Grabber}
% \WorkingDrawing{Suspension}
% \WorkingDrawing{Cockpit}
% \WorkingDrawing{Joystick}
% \WorkingDrawing{MechanicalDisplay}

\section{Working Drawings}
A product cannot be counted as finished unless there are plans to manufacture that product. While the plans for our product is definitely beyond the ability of a student run organization (or small countries), 
we've added working drawings to highlight important features and dimensions of our work.

\newcommand{\WorkingDrawing}[1]{
 \subsection{#1}
 \input{a-1-4-ManufacturingWorkingDrawing/b-1-WorkingDrawing/c-#1/#1.tex}
 }


\clearpage
\section{Exploded View} \label{sec:ExplodedView}
 % 

\clearpage
\section{ExplodedView}

Insert spiel about ExplodedView.
\clearpage

\subsection{Antenna}
 \WorkingDrawing{a-1-4-ManufacturingWorkingDrawing/b-1-WorkingDrawing/c-Antenna/Rodriguez_Juan_Signal_Bar.JPG}{Signal Bar}
\WorkingDrawing{a-1-4-ManufacturingWorkingDrawing/b-1-WorkingDrawing/c-Antenna/Rodriguez_Juan_Swivel_link.JPG}{Swivel Link}
\WorkingDrawing{a-1-4-ManufacturingWorkingDrawing/b-1-WorkingDrawing/c-Antenna/Rodriguez_Juan_antenna_assembly.JPG}{Antenna Assembly}
\WorkingDrawing{a-1-4-ManufacturingWorkingDrawing/b-1-WorkingDrawing/c-Antenna/Rodriguez_Juan_antenna_base.JPG}{Antenna Base}
\WorkingDrawing{a-1-4-ManufacturingWorkingDrawing/b-1-WorkingDrawing/c-Antenna/Rodriguez_Juan_antenna_exploded.JPG}{Antenna Exploded}
\WorkingDrawing{a-1-4-ManufacturingWorkingDrawing/b-1-WorkingDrawing/c-Antenna/Rodriguez_Juan_antenna_support.JPG}{Antenna Support}
\WorkingDrawing{a-1-4-ManufacturingWorkingDrawing/b-1-WorkingDrawing/c-Antenna/Rodriguez_Juan_crossbar.JPG}{Crossbar}
\WorkingDrawing{a-1-4-ManufacturingWorkingDrawing/b-1-WorkingDrawing/c-Antenna/Rodriguez_Juan_pin_a.JPG}{Pin A}
\WorkingDrawing{a-1-4-ManufacturingWorkingDrawing/b-1-WorkingDrawing/c-Antenna/{Rodriguez_Juan_pin_b.JPG}{Pin B}
\WorkingDrawing{a-1-4-ManufacturingWorkingDrawing/b-1-WorkingDrawing/c-Antenna/Rodriguez_Juan_real_antenna.JPG}{Real Antenna}
\WorkingDrawing{a-1-4-ManufacturingWorkingDrawing/b-1-WorkingDrawing/c-Antenna/Rodriguez_Juan_receiver.JPG}{Receiver}

\clearpage

\subsection{Lights}
 \newcommand{\DetailDesignLights}[2]{\drawing{/a-1-3-DetailDesign/b-Lights/#1}{Sackett, Justin: #2}
\DetailDesignDrawing{/a-1-3-DetailDesign/b-Lights/LIGHTBASE.JPG}{\justin Light Base}
\DetailDesignDrawing{/a-1-3-DetailDesign/b-Lights/LIGHTSCREWS.JPG}{\justin Light Screws}
\DetailDesignDrawing{/a-1-3-DetailDesign/b-Lights/LED_Work_light_27w.JPG}{\justin LED Work Light}

\clearpage


\clearpage
\section{Part List} \label{sec:PartList} 
% \newcommand{\PartListDrawing}[2]{
\begin{figure}[!ht]
    \centering
    \includegraphics[height=0.85\textheight]{#1}
    \caption{#2}
\end{figure}
}

\subsection{Antenna}
\WorkingDrawing{a-1-4-ManufacturingWorkingDrawing/b-1-WorkingDrawing/c-Antenna/Rodriguez_Juan_Signal_Bar.JPG}{Signal Bar}
\WorkingDrawing{a-1-4-ManufacturingWorkingDrawing/b-1-WorkingDrawing/c-Antenna/Rodriguez_Juan_Swivel_link.JPG}{Swivel Link}
\WorkingDrawing{a-1-4-ManufacturingWorkingDrawing/b-1-WorkingDrawing/c-Antenna/Rodriguez_Juan_antenna_assembly.JPG}{Antenna Assembly}
\WorkingDrawing{a-1-4-ManufacturingWorkingDrawing/b-1-WorkingDrawing/c-Antenna/Rodriguez_Juan_antenna_base.JPG}{Antenna Base}
\WorkingDrawing{a-1-4-ManufacturingWorkingDrawing/b-1-WorkingDrawing/c-Antenna/Rodriguez_Juan_antenna_exploded.JPG}{Antenna Exploded}
\WorkingDrawing{a-1-4-ManufacturingWorkingDrawing/b-1-WorkingDrawing/c-Antenna/Rodriguez_Juan_antenna_support.JPG}{Antenna Support}
\WorkingDrawing{a-1-4-ManufacturingWorkingDrawing/b-1-WorkingDrawing/c-Antenna/Rodriguez_Juan_crossbar.JPG}{Crossbar}
\WorkingDrawing{a-1-4-ManufacturingWorkingDrawing/b-1-WorkingDrawing/c-Antenna/Rodriguez_Juan_pin_a.JPG}{Pin A}
\WorkingDrawing{a-1-4-ManufacturingWorkingDrawing/b-1-WorkingDrawing/c-Antenna/{Rodriguez_Juan_pin_b.JPG}{Pin B}
\WorkingDrawing{a-1-4-ManufacturingWorkingDrawing/b-1-WorkingDrawing/c-Antenna/Rodriguez_Juan_real_antenna.JPG}{Real Antenna}
\WorkingDrawing{a-1-4-ManufacturingWorkingDrawing/b-1-WorkingDrawing/c-Antenna/Rodriguez_Juan_receiver.JPG}{Receiver}


\subsection{Lights}
\newcommand{\DetailDesignLights}[2]{\drawing{/a-1-3-DetailDesign/b-Lights/#1}{Sackett, Justin: #2}
\DetailDesignDrawing{/a-1-3-DetailDesign/b-Lights/LIGHTBASE.JPG}{\justin Light Base}
\DetailDesignDrawing{/a-1-3-DetailDesign/b-Lights/LIGHTSCREWS.JPG}{\justin Light Screws}
\DetailDesignDrawing{/a-1-3-DetailDesign/b-Lights/LED_Work_light_27w.JPG}{\justin LED Work Light}


\subsection{Suspension}
\AssemblyManualStep{1}{a-1-4-ManufacturingWorkingDrawing/b-2-AssemblyInstructionManual/c-Suspension/step1.PNG}{
Assemble Full Wheel by taking wheel and putting into Tire.
}{\asimm Assembly Step 1}

\AssemblyManualStep{2}{a-1-4-ManufacturingWorkingDrawing/b-2-AssemblyInstructionManual/c-Suspension/step2.PNG}{
Insert Hub into Wheel
}{\asimm Assembly Step 2}

\AssemblyManualStep{3}{a-1-4-ManufacturingWorkingDrawing/b-2-AssemblyInstructionManual/c-Suspension/step3.PNG}{
Mount Brake Disk and Brake Caliper onto Hub
}{\asimm Assembly Step 3}

\AssemblyManualStep{4}{a-1-4-ManufacturingWorkingDrawing/b-2-AssemblyInstructionManual/c-Suspension/step4.PNG}{
Mount hub and wheel assembly onto upright
}{\asimm Assembly Step 4}

\AssemblyManualStepTwo{5}{a-1-4-ManufacturingWorkingDrawing/b-2-AssemblyInstructionManual/c-Suspension/step5.PNG}{
Mount lower control arms and upper controls arms onto frame. Mount coilover onto LCA. Mount upright between control arms. Mirror assembly accross the frame. 
}{\asimm Assembly Step 5}


\subsection{Grabbers}
\drawing{/a-1-3-DetailDesign/b-1-WorkingDrawing/c-Antenna/Rodriguez_Juan_Signal_Bar.JPG}{Rodriguez, Juan: Chasis Render}


\subsection{Emergency Button}
\ExplodedViewDrawing{a-1-4-ManufacturingWorkingDrawing/b-3-ExplodedView/c-EmergencyButton/EmergencyButton.JPG}{\vishakh Exploded View of Emergency Button}


\subsection{Mechanical Display}
\DetailDesignDrawing{a-1-3-DetailDesign/b-MechanicalDisplay/MechanicalDisplay.png}{\vishakh Mechanical Display View 1}
\DetailDesignDrawing{a-1-3-DetailDesign/b-MechanicalDisplay/MechanicalDisplay1.png}{\vishakh Mechanical DisplayView 2}
\DetailDesignDrawing{a-1-3-DetailDesign/b-MechanicalDisplay/MechanicalDisplay2.png}{\vishakh Mechanical Display View 3}
%\DetailDesignDrawing{a-1-3-DetailDesign/b-MechanicalDisplay/MechanicalDisplay3.png}{\vishakh Mechanical Display View 4}


